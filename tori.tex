\documentclass{amsart}

\usepackage{amscd}
\usepackage{amssymb, amsfonts}
\usepackage[notcite, notref]{showkeys}
\usepackage{amsrefs}
\input xy
\xyoption{all}
%\input{Gdefinitions3.tex}
\usepackage{color}
\usepackage[T1]{fontenc}
\usepackage{tikz}
%\usetikzlibrary{shapes,arrows,calc,matrix}
\usepackage{tikz-cd}
\usepackage{mathrsfs}

\definecolor{bettergreen}{rgb}{0,.7,0}
\newcommand\blue[1]{{\color{blue}{#1}}}
\newcommand\green[1]{{\color{bettergreen}{#1}}}
\newcommand\red[1]{{\color{red}{#1}}}

%\long\def\comment#1{\marginpar{{\footnotesize\color{red} #1\par}}}
%\long\def\commentimmi#1{\marginpar{{\footnotesize\color{bettergreen} #1\par}}}
%\long\def\change#1{{\color{blue} #1}}
%newcommand\green[1]{{\color{green} #1}}
%\newcommand   [1]{{\color{red}\small #1}}

%\let\immi=\green

%\let\raf=\blue


%%%% Definitions of things you already have %%%%%%%%%%
\newcommand{\Q}{{\mathbb Q}}
\newcommand{\C}{{\mathbb C}}
\newcommand{\R}{{\mathbb R}}
\newcommand{\F}{{\mathbb F}}
\newcommand{\Z}{{\mathbb Z}}
\newcommand{\N}{{\mathbb N}}
%\newcommand{\bA}{{\mathbb A}}
\newcommand{\LL}{{\mathbb L}}

\def\CC{{\mathbb C}}
\def\ZZ{{\mathbb Z}}
\newcommand{\cF}{\mathcal{F}}
\newcommand{\cO}{\mathcal{O}}


\newcommand{\bF}{\mathbf{F}}
\newcommand{\ri}{\mathcal{O}}
\newcommand{\gl}{\mathfrak{gl}}
\newcommand{\GL}{\mathbf {GL}}
\newcommand{\fg}{\mathfrak{g}}
\newcommand{\ft}{\mathfrak{t}}
\newcommand{\fz}{\mathfrak{z}}
\newcommand{\fsp}{\mathfrak{sp}}
\newcommand{\fsl}{\mathfrak{sl}}
\newcommand{\Ad}{\operatorname{Ad}}
\newcommand{\jac}{\operatorname{Jac}}
\newcommand{\gal}{\operatorname{Gal}}
\newcommand{\res}{\operatorname{Res}}
\newcommand{\vol}{\operatorname{vol}}
\newcommand{\loc}{\operatorname{Loc}}

\newcommand{\aut}{\mathbf{Aut}}
\newcommand{\reg}{\mathrm{reg}}
%\newcommand{\spl}{\mathrm{sp}}
\newcommand{\ur}{\mathrm{ur}}


\newcommand{\ep}{\operatorname{EP}}
\newcommand{\cM}{\mathcal {M}}
\newcommand{\cL}{\mathcal {L}}

\newcommand{\ram}{\mathrm{Ram}}
%\newcommand{\sem}{\mathrm{ss}}
\newcommand{\A}{\mathbb{A}}

%\newcommand{\bG}{\mathbf {G}}
\newcommand{\bG}{\mathbf{G}}
\newcommand{\bT}{\mathbf {T}}
\newcommand{\bS}{\mathbf{S}}
\newcommand{\bR}{\mathbf{R}}
\newcommand{\bM}{\mathbf {M}}
\newcommand{\cV}{\mathcal{V}}
\newcommand{\can}{\mathrm{can}}
\newcommand{\cG}{\mathcal{G}}
\newcommand{\ff}{{\mathfrak f}}

%\newcommand{\ug}{{\mathrm U}}
%\newcommand{\fs}{{\mathfrak s}}

\newcommand{\oi}{{\bf \mathrm{O}}}


\newcommand{\rf}{k}

%%%% Motivic  definitions %%%%%%%

\newcommand{\ldp}{{\mathcal L}_{\mathrm {DP}}}
\newcommand\ldpo[1][\ri]{{\mathcal L}_{#1}}
\newcommand\cmf[1]{{\mathcal C}(#1)}
\newcommand\cA{{\mathcal A}}
%\newcommand\co{{\mathcal O}}
\newcommand\cB{{\mathcal B}}
\newcommand\ac{\overline{\mathrm{ac}}}
\newcommand\lef{\mathbb L}
\newcommand\cP{{\mathcal P}}
\newcommand\cC{{\mathcal C}}
\newcommand\cE{{\mathcal E}}
\newcommand\cX{{\mathcal X}}
%\newcommand\bT{{\mathbf T}}
\newcommand\mot{\mathrm{mot}}
\newcommand\spl{\mathrm{spl}}
\newcommand{\scD}{\mathscr{D}}


\newcommand{\de}{{\text{Def}}}
\newcommand{\rde}{{\text{RDef}}}
\newcommand{\K}{F}
\newcommand{\mexp}{\mathbf{e}}
\newcommand{\Ner}[1]{\mathcal{#1}}
\newcommand{\NerC}[1]{\mathcal{#1}^\circ}

\def\llp{\mathopen{(\!(}}
\def\llb{\mathopen{[\![}}
\def\rrp{\mathopen{)\!)}}
\def\rrb{\mathopen{]\!]}}

\DeclareMathOperator{\Gal}{Gal}
\DeclareMathOperator{\Ind}{Ind}
\DeclareMathOperator{\ord}{ord}
\DeclareMathOperator{\Res}{Res}
\DeclareMathOperator{\Hom}{Hom}

%%%%%%%%%%%%% Theorem declarations %%%%%%%%%%%%%
\theoremstyle{plain}
\newtheorem{thm}{Theorem}
\newtheorem{theorem}[thm]{Theorem}
\newtheorem{lem}[thm]{Lemma}
\newtheorem{cor}[thm]{Corollary}
%\newtheorem{defn}[thm]{Definition}
%\newtheorem{rem}[thm]{Remark}
\newtheorem{prop}[thm]{Proposition}

\theoremstyle{definition}
\newtheorem{rem}[thm]{Remark}
\newtheorem{defn}[thm]{Definition}
\newtheorem{example}[thm]{Example}


\title[]{Motivic measures and the motive of a reductive group}
\author{Julia Gordon and David Roe}

\begin{document}

\maketitle

\section{Introduction}
The goal of this paper is to complete a technical step in the 
``motivic" formulation of 
the representation theory of $p$-adic groups, that was started by T. Hales in 1999. 
Here  the word ``motivic'' refers to the use of the theory of motivic integration, as developed by R. Cluckers and F. Loeser in \cite{cluckers-loeser}; in this paper, no integration will be required, so ``definable'' would have been a better term to use. 

Specifically, our goal is to prove that the canonical maximal parahoric subgroup of a connected reductive $p$-adic group, as defined by  B. Gross in \cite{gross:motive} using Bruhat-Tits theory, is definable using Denef-Pas language, which is the language used in Cluckers-Loeser theory of motivic integration and, consequently,  in all its applications to representation theory of $p$-adic groups. As a consequence, we  prove that for any connected reductive group the canonical Haar measure (in the sense of \cite{gross:motive}) is motivic.
{\bf XX Can we think of the motive of a reductive group as its motivic volume? Careful though: Artin-Tate motives vs. Chow motives. think about it.}

In fact, for unramified groups this statement has been known for a while, cf. \cite{cluckers-hales-loeser}. However, for a ramified group $\bG$, there is a difficulty caused by the fact that the canonical smooth model of $\bG$, whose definition relies on the N\'eron model of a maximal torus in $\bG$, does not behave well with respect to Galois descent. Thus, the main technical difficulty we deal with in this paper is proving that the connected component of the N\`eron model of a tamely ramified torus is definable in the language of Denef-Pas. We note that this difficulty is largely caused by the fact that ``taking the connected component'' is not, naturally, an operation that can easily be described by first-order logic. 

{\bf XX Fill in: application to the formal degree ?}

\section{Tori} 
In this paper, we will use the notions of a definable set and  definable function. These refer to being definable in Denef-Pas language. 
Formulas in Denef-Pas language can have variables of three \emph{sorts}: valued field (which will be denoted by $VF$), residue field (denoted by $RF$) and the value group. Even though we will be often be working with ramified extensions, we always start with a local field $F$ with normalized valuation, so the value group is $\Z$ (the $VF$-variables will range over $F$, and so their valuations will be in $\Z$).
Formulas in Denef-Pas language can be interpreted given a choice of a valued field \emph{together with a uniformizer}. 
We refer the reader to \cite{what's the best ref?} for the definitions of Denef-Pas language, definable sets, etc. 

For us, $F$ will always be a non-Archimedean local field: either $\F_q\llp t\rrp$ or a finite extension of $\Q_p$.
As a consequence of the definition of a definable set, all statements in this paper will hold for any $F$ of sufficiently large residue characteristic $p$, 
though we will give no effective bound on $p$. 
Given an integer $M>0$, we will denote by $\loc_M$ the collection of non-Archimedean local fields that are finite extensions of $\Q_p$ or $\F_p((t))$ with residue characteristic greater than $M$. 

Since we have to assume that $p$ is large relative to some fixed parameters, such as the degree of the extension $E/F$ over which our torus splits, we may assume that $E/F$ is tame.  

For such a local field $F$, we will denote its ring of integers by $\ri_F$, its residue field by $k_F$, and let $q_F=\# k_F$. The symbol $\varpi_F$ will always stand for the uniformizer of the valuation on $F$; when there is no possibility of confusion, we might drop the subscript $F$. 
A formula in Denef-Pas language  with $n$ free $VF$-variables, $m$ free $RF$-variables, and $r$ free 
$\Z$-variables 
defines a subset of $F^n\times k_F^m \times \Z^r$. 
We will denote the definable set $F^n\times k_F^m \times \Z^r$ itself by $VF^n\times RF^m\times \Z^r$. In earlier works on motivic integration this set was denoted by $h[n,m,r]$. 

We start by setting up the framework for working with tori in the Denef-Pas language, following \cite{cluckers-hales-loeser}, \cite{CGH-2} and \cite{hales:transfert}.



\subsection{Fixed choices}

As in \cite{hales:transfert}*{\S 2.1}, we begin by outlining our \emph{fixed choices}, which are made before writing any formulas in the Denef-Pas language.  
In this way, we split the definition of the torus $\bT$ into choices made without reference to the base field $F$ and formulas determining the dependence on $F$.

We fix a finite group $\Gamma$ and a normal subgroup $I \unlhd \Gamma$, as well as enumerations of their elements $\Gamma = \{\sigma_1, \dots, \sigma_m\}$ and $I = \{\sigma_1, \dots, \sigma_e\}$.  We make the convention that $\sigma_1 = 1$ and $\sigma_m$ generates $\Gamma / I$.  These groups will play the role of $\Gal(E/F)$ and its inertia subgroup, where $E$ is the splitting field of $\bT$.

In order to define a torus $\bT$, we will use the equivalence of categories between $F$-tori and free $\Z$-modules with a Galois action.  To this end, we fix a positive integer $n$ and an injective homomorphism
\[
\theta : \Gamma \hookrightarrow \GL_n(\Z),
\]
which gives $\Z^n$ an action of $\Gamma$.  The $\Gamma$-module $X$ defined by $\theta$ will play the role of $X_\ast(\bT)$.

Finally, we fix a resolution of $X$ by an induced $\Gamma$-module $Y$, i.e. a surjective map $Y \to X$ where $Y$ has a basis permuted by $\Gamma$ (c.f. \cite{brahm}*{Satz 0.4.4}).  This resolution will allow us to definably cut out the connected component of the N\'eron model inside $\bT(F)$.

%an abstract split torus ${\bT}^\spl$ and an isomorphism $X^\ast(\bT^\spl) \cong \Z^n$.
%\footnote{in contrast to \cite{cluckers-hales-loeser}, where they parameterize $X^\ast(\bT^\spl)$ using valued field variables, since they need characters as functions to $F^\times$.}
%In order to define $\bT$ as a twist of $\bT^\spl$, we fix a homomorphism


\subsection{Parameterizing field extensions and tori}
 
We encode field extensions in the same way as in \cite{CGH-2}.
Namely, we parameterize Galois extensions $E/F$ with $\Gal(E/F) \cong \Gamma$
and realize all tori over $F$ that split over $E$ with character lattice $X$.
This parameterizes such tori as members of a family of definable sets,
for all $F$ of sufficiently large residue characteristic. 

We will write $L$ for the unramified subextension of $E/F$.  In order to encode the data of the extension tower $E/L/F$, we let $f=m/e$ and introduce parameters $b_0,\dots, b_{f-1}$, ranging over $\ri_F$.
We set $b(x)=x^f+b_{f-1}x^{f-1}+ \dots + b_0$. 
Similarly, we introduce parameters $c_0, \dots, c_{e-1}$, ranging over $L$
(i.e., each is given by an $f$-tuple of elements of $F$) and set $c(y) = y^e + c_{e-1}y^{e-1} + \dots + c_0$.

We impose the following conditions on these parameters, all of which are definable by formulas in the Denef-Pas language. 
\begin{enumerate}
\item The reduction of $b(x)$ modulo $\varpi_F$ is irreducible over $k_F$. 
This ensures that $F[x]/(b(x))$ is a degree $f$ unramified field extension of $F$. 
We denote this extension by $L$, and once and for all fix an identification with $F^f$ as 
an $F$-vector space. 
\item The polynomial $c(x)$ is Eisenstein: $\ord_L(c_0) = 1$ and $\ord_L(c_i) \ge 1$ for all $i$.
We further assume that the resulting extension $E = L[x]/(c(x))$ is Galois over $F$.
We fix an identification of $E$ with $L^e$ as $L$-vector spaces, and thus with $F^m$ as $F$-vector spaces. 
\item The field automorphisms of $E$ over $F$, as specified by $m \times m$ matrices over $F$, form a group isomorphic to $\Gamma$.
We will write $\sigma_i$ for the matrix as well as the corresponding element of $\Gamma$.
\item The automorphisms $\sigma_1, \dots, \sigma_e$ fix $L$, and the restriction of $\sigma_m$ to $L$ has order $f$.
\end{enumerate}
We denote by $\cE_\Gamma$ the space of parameters $(b_0, \dots, b_{f-1}, c_0, \dots, c_{e-1}, \sigma_1, \dots, \sigma_m)$ with these properties,
thought of as a definable subset of some large affine space over $F$.
For each local field $F$, every element of $\cE_\Gamma$ gives rise to a tower of field extensions $E/L/F$
% Note that we don't actually need to require that F has large residue characteristic in this claim.
with $\gal(E/F)$ isomorphic to $\Gamma$ and satisfying all the above conditions.
The homomorphism $\theta$ then defines a torus $\bT$ over $F$ with cocharacter lattice $X$.
This torus splits over $E$, with $\gal(E/F) \cong \Gamma$ acting on $\bT(E) = E^\times \otimes X$ diagonally.
This action is definable, and therefore the set $\bT(F) = \bT(E)^\Gamma$ is as well.

Note that different parameters in $\cE_\Gamma$ may yield isomorphic extensions, but that all isomorphism classes of $E/F$ with Galois group $\Gamma$
arise from some element of $\cE_\Gamma$.  Moreover, as $\theta$ ranges over all homomorphisms $\Gamma \hookrightarrow \GL_n(\Z)$, all tori of dimension $n$ with splitting field $E$ will appear.
\begin{example}
Suppose $\Gamma = I = \Z / 2\Z$ and $n=1$; note that $\theta$ is uniquely determined in this case.
For $p > 2$, the two ramified quadratic extensions of $F$ appear as members of the same family,
one for the polynomial $c(x) = x^2-\varpi$, another for the polynomial $c(x) = x^2-\epsilon\varpi$,
where $\epsilon \in \ri_F^\times$ is a non-square.
Recall that the interpretation of formulas in Denef-Pas language depends not just on the field,
but also on the choice of uniformizer. In this case, a different choice of the uniformizer
would cause these two extensions to switch places, but they would still both appear.
The torus $\bT$ is the the one-dimensional unitary group that splits over $E$.
\end{example}
%\begin{rem}
%We observe that in this context we cannot use standard number-theoretic constructions involving objects such as the cyclotomic polynomial, since we need to be able to let $F$ with its residue characteristic vary, and thus polynomials whose degree depends on $p$ are not allowed. For the same reason, we cannot use the %Frobenius automorphism.  
%\end{rem}

\subsection{The identity component of the N\'eron model}

With $E/L/F$ and $\bT$ parameterized and $\bT(F)$ definable, we may now prove the main technical result of the paper.
Write $\Ner{T}$ for the N\'eron model of $\bT$ (c.f. \cite{bosch-lutkebohmert-raynaud:neron}*{Ch. 10}); this is a model for $\bT$ over $\ri_F$ with the property that $\Ner{T}(\ri_F) = \bT(F)$.  Let $\NerC{T}$ be its identity component.

\begin{prop}
The subset $\NerC{T}(\ri_F) \subseteq \bT(F)$ is definable.
\end{prop}
\begin{proof}
We first consider the case that $L = F$.  Then the identity component $\NerC{T}(\ri_F)$ is the kernel of the map $w_\bT : \bT(L) \to X_I$ from $\Ner{T}(\ri_L)$ to its component group (c.f. \citelist{\cite{bitan}*{3.1} \cite{kottwitz:isocrystals-2}*{\S 7}}).

Our fixed choice of resolution $Y \to X$ yields an induced torus $\bR$ over $L$ with cocharacter lattice $Y$, together with a diagram
\[
\begin{tikzcd}
\bR(L) \rar{\alpha} \dar{w_{\bR}} \drar{\beta} & \bT(L) \rar \dar{w_{\bT}} & 1 \\
Y_I \rar & X_I \rar & 0
\end{tikzcd}
\]
as in \cite{kottwitz:isocrystals-2}*{(7.2.6)}.  The map $\bR(L) = (E^\times \times Y)^I \to (E^\times \times X)^I = \bT(L)$
is definable since it is induced by the fixed map $Y \to X$.  Since $\bR$ is induced,
$Y_I$ is torsion free and $w_\bR : \bR(L) \to Y_I = \Hom(X^\ast(\bR)$ is given by
$r \mapsto \left(\lambda \mapsto \ord_{L}(\lambda(r))\right)$ \cite{kottwitz:isocrystals-2}*{(7.2.3)}.
Therefore $w_\bR$ is definable, and so is the composition $\beta : \bR(L) \to X_I$ with the fixed map $Y_I \to X_I$.

We may now show that $\NerC{T}(\ri_L)$ is a definable subset of $\bT(L)$: we have $t \in \NerC{T}(\ri_L)$ if and only if $\exists r \in \bR(L)$ such that $\alpha(r) = t$ and $\beta(r) = 0$.  Finally, $\NerC{T}(\ri_F) = \NerC{T}(\ri_L)^\Gamma$ is definable since it consists of the points of $\NerC{T}(\ri_L)$ fixed under the action of $\Gamma$.
\end{proof}

\section{General reductive groups}
%{\bf XX Here want to prove that $G_{x, 0}$ is definable; canonical measure is motivic; formal degree is motivic. }
It is shown in \cite{CGH-2} that for all $r>0$ and any \emph{optimal} point $x$ in the building, the Moy-Prasad filtration subgroups $G_{x, r}$ are definable; here we fill in the last missing case, namely, we prove that $G_{\ff}:=G_{x, 0}$ (with $x$ in the interior of the facet $\ff$)   is definable for  all facets $\ff$.  

We treat reductive groups in the definable setting as in \cite{hales:transfert} and \cite{CGH-2}.
In particular, we have the \emph{fixed choices} that include the absolute root datum of $G$ and the root datum of $G$ over the maximal unramified extension; the Galois action on the absolute root datum, and, finally,  a finite set that encodes the set of faces of an alcove in the building of $G$ over the maximal unramified extension.

We recall that each  fixed choice determines  a split reductive group $G^{\ast\ast}$ and an embedding 
$G^{\ast\ast}(F)\hookrightarrow GL_N(F)$ for every local field $F$ of sufficiently large residue characteristic.  
Moreover, as in \cite{hales:transfert}*{\S 2.2}, we  have a parameter space $Z$ that encodes field extensions and Galois cocycles, so that a pair 
$(\Sigma, z)$ determines a connected reductive group $G(F)$ for every local field $F$ of sufficiently large residue characteristic, and every reductive group (up to isomorphism)  arises via this construction. 
(We recall that here $\Sigma$ is a fixed choice determining $G^{\ast\ast}$, and $z$ is a parameter that ranges over $F^m$ for some $m$ and determines a field extension over which $G$ splits, and the Galois cocycle that corresponds to $G$). 

Our main result is that 
the compact subgroup $G(F)_\ff$ associated with $\ff$ via Bruhat-Tits theory is definable.  

%which a reductive group with a given definable family of definable sets $G(F)$ that

\begin{prop} Let $G$ be a tamely ramified quasi-split connected reductive group defined over a local field $F$. Let $\ff$ be a facet in the building of $G$. 
Then $G(F)_\ff\subset G(F)$ arises in a definable family of definable sets. 
In particular, the canonical parahoric of $G$ defined in \cite{gross:motive} is definable. 
\end{prop}

\begin{proof} Let $T$ be a maximal torus containing the maximal split torus in $G$; let $x$ be the baricentre of $\ff$.
By definition, $G_x$ is generated by $T_x$ and $U_\psi$, where $U_\psi$ are the filtration subgroups of the unipotent one-parameter subgroups $U_{\alpha}$. 
We have shown in Proposition \ref{} above that  $T(F)_x$ is definable. The rest of the proof is identical to that of Lemma 3.4 in \cite{CGH-2}.  
\end{proof} 

As an immediate consequence, we obtain that the canonical measure is motivic, up to a motivic constant. 
We recall that a \emph{motivic constant} is an element of the ring of constructible motivic functions on a point, i.e.,  of $A:=\Z[\lef, \lef^{-1}, \frac{1}{1-\lef^{-i}}, i>0]$. 
This statement was previously known for unramified reductive groups, \cite{cluckers-hales-loeser}. 

For a connected reductive group $G$ over a local field $F$, we denote the \emph{canonical Haar measure} on $G(F)$ in the sense of \cite{gross:motive} by $d\mu_G^\can$. 
\begin{theorem} For every fixed choice $\Psi$ of a root datum there exists $M>0$ that depends only on 
$\Psi$, a  definable set $\bG \to Z$, a family of  motivic measures $d\mu_z$ on $G_z$, and a motivic function $c:Z\to A$, such that for every $F\in \loc_M$: 
\begin{itemize} 
\item $G_z(F)$ is a connected reductive group over $F$ with root datum $\Psi$;
\item $c_F(z) d\mu_{G_z}^\can = d\mu_{z, F}$.
 \end{itemize} 
\end{theorem} 




\begin{bibdiv}
\begin{biblist}
\bibselect{references}
\end{biblist}
\end{bibdiv}

\end{document}





























