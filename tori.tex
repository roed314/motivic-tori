\documentclass{amsart}

\usepackage{amscd}
\usepackage{amssymb, amsfonts}
\usepackage[notcite, notref]{showkeys}
\usepackage{amsrefs}
\input xy
\xyoption{all}
%\input{Gdefinitions3.tex}
\usepackage{color}
\usepackage[T1]{fontenc}
\usepackage{tikz}
\usepackage{mathrsfs}

\definecolor{bettergreen}{rgb}{0,.7,0}
\newcommand\blue[1]{{\color{blue}{#1}}}
\newcommand\green[1]{{\color{bettergreen}{#1}}}
\newcommand\red[1]{{\color{red}{#1}}}

%\long\def\comment#1{\marginpar{{\footnotesize\color{red} #1\par}}}
%\long\def\commentimmi#1{\marginpar{{\footnotesize\color{bettergreen} #1\par}}}
%\long\def\change#1{{\color{blue} #1}}
%newcommand\green[1]{{\color{green} #1}}
%\newcommand   [1]{{\color{red}\small #1}}

%\let\immi=\green

%\let\raf=\blue


%%%% Definitions of things you already have %%%%%%%%%%
\newcommand{\Q}{{\mathbb Q}}
\newcommand{\C}{{\mathbb C}}
\newcommand{\R}{{\mathbb R}}
\newcommand{\F}{{\mathbb F}}
\newcommand{\Z}{{\mathbb Z}}
\newcommand{\N}{{\mathbb N}}
%\newcommand{\bA}{{\mathbb A}}
\newcommand{\LL}{{\mathbb L}}

\def\CC{{\mathbb C}}
\def\ZZ{{\mathbb Z}}
\newcommand{\cF}{\mathcal{F}}
\newcommand{\cO}{\mathcal{O}}


\newcommand{\bF}{\mathbf{F}}
\newcommand{\ri}{\mathcal{O}}
\newcommand{\gl}{\mathfrak{gl}}
\newcommand{\GL}{\mathbf {GL}}
\newcommand{\fg}{\mathfrak{g}}
\newcommand{\ft}{\mathfrak{t}}
\newcommand{\fz}{\mathfrak{z}}
\newcommand{\fsp}{\mathfrak{sp}}
\newcommand{\fsl}{\mathfrak{sl}}
\newcommand{\Ad}{\operatorname{Ad}}
\newcommand{\jac}{\operatorname{Jac}}
\newcommand{\gal}{\operatorname{Gal}}
\newcommand{\res}{\operatorname{Res}}
\newcommand{\vol}{\operatorname{vol}}
\newcommand{\aut}{\mathbf{Aut}}
\newcommand{\reg}{\mathrm{reg}}
%\newcommand{\spl}{\mathrm{sp}}
\newcommand{\ur}{\mathrm{ur}}


\newcommand{\ep}{\operatorname{EP}}
\newcommand{\cM}{\mathcal {M}}
\newcommand{\cL}{\mathcal {L}}

\newcommand{\ram}{\mathrm{Ram}}
%\newcommand{\sem}{\mathrm{ss}}
\newcommand{\A}{\mathbb{A}}

%\newcommand{\bG}{\mathbf {G}}
\newcommand{\bG}{\mathbf{G}}
\newcommand{\bT}{\mathbf {T}}
\newcommand{\bS}{\mathbf{S}}
\newcommand{\bR}{\mathbf{R}}
\newcommand{\bM}{\mathbf {M}}
\newcommand{\cV}{\mathcal{V}}
\newcommand{\can}{\mathrm{can}}
\newcommand{\cG}{\mathcal{G}}

%\newcommand{\ug}{{\mathrm U}}
%\newcommand{\fs}{{\mathfrak s}}

\newcommand{\oi}{{\bf \mathrm{O}}}


\newcommand{\rf}{k}

%%%% Motivic  definitions %%%%%%%

\newcommand{\ldp}{{\mathcal L}_{\mathrm {DP}}}
\newcommand\ldpo[1][\ri]{{\mathcal L}_{#1}}
\newcommand\cmf[1]{{\mathcal C}(#1)}
\newcommand\cA{{\mathcal A}}
%\newcommand\co{{\mathcal O}}
\newcommand\cB{{\mathcal B}}
\newcommand\ord{\mathrm{ord}}
\newcommand\ac{\overline{\mathrm{ac}}}
\newcommand\lef{\mathbb L}
\newcommand\cP{{\mathcal P}}
\newcommand\cC{{\mathcal C}}
\newcommand\cE{{\mathcal E}}
\newcommand\cX{{\mathcal X}}
%\newcommand\bT{{\mathbf T}}
\newcommand\mot{\mathrm{mot}}
\newcommand\spl{\mathrm{spl}}
\newcommand{\scD}{\mathscr{D}}
\newcommand{\Res}{\mathrm{Res}}


\newcommand{\de}{{\text{Def}}}
\newcommand{\rde}{{\text{RDef}}}
\newcommand{\K}{F}
\newcommand{\mexp}{\mathbf{e}}
\newcommand{\Ner}[1]{\mathcal{#1}}
\newcommand{\NerC}[1]{\mathcal{#1}^\circ}

\def\llp{\mathopen{(\!(}}
\def\llb{\mathopen{[\![}}
\def\rrp{\mathopen{)\!)}}
\def\rrb{\mathopen{]\!]}}

\DeclareMathOperator{\Gal}{Gal}
\DeclareMathOperator{\Ind}{Ind}


%%%%%%%%%%%%% Theorem declarations %%%%%%%%%%%%%
\theoremstyle{plain}
\newtheorem{thm}{Theorem}
\newtheorem{theorem}[thm]{Theorem}
\newtheorem{lem}[thm]{Lemma}
\newtheorem{cor}[thm]{Corollary}
%\newtheorem{defn}[thm]{Definition}
%\newtheorem{rem}[thm]{Remark}
\newtheorem{prop}[thm]{Proposition}

\theoremstyle{definition}
\newtheorem{rem}[thm]{Remark}
\newtheorem{defn}[thm]{Definition}
\newtheorem{example}[thm]{Example}


\title[]{Motivic measures and the motive of a reductive group}
\author{Julia Gordon and David Roe}

\begin{document}

\maketitle

\section{Introduction}
The goal of this paper is to complete a technical step in the 
``motivic" formulation of 
the representation theory of $p$-adic groups, that was started by T. Hales in 1999. 
Here  the word ``motivic'' refers to the use of the theory of motivic integration, as developed by R. Cluckers and F. Loeser in \cite{cluckers-loeser}; in this paper, no integration will be required, so ``definable'' would have been a better term to use. 

Specifically, our goal is to prove that the canonical maximal parahoric subgroup of a connected reductive $p$-adic group, as defined by  B. Gross in \cite{gross:motive} using Bruhat-Tits theory, is definable using Denef-Pas language, which is the language used in Cluckers-Loeser theory of motivic integration and, consequently,  in all its applications to representation theory of $p$-adic groups. As a consequence, we  prove that for any connected reductive group the canonical Haar measure (in the sense of \cite{gross:motive}) is motivic.
{\bf XX Can we think of the motive of a reductive group as its motivic volume? Careful though: Artin-Tate motives vs. Chow motives. think about it.}

In fact, for unramified groups this statement has been known for a while, cf. \cite{cluckers-hales-loeser}. However, for a ramified group $\bG$, there is a difficulty caused by the fact that the canonical smooth model of $\bG$, whose definition relies on the N\'eron model of a maximal torus in $\bG$, does not behave well with respect to Galois descent. Thus, the main technical difficulty we deal with in this paper is proving that the connected component of the N\`eron model of a tamely ramified torus is definable in the language of Denef-Pas. We note that this difficulty is largely caused by the fact that ``taking the connected component'' is not, naturally, an operation that can easily be described by first-order logic. 

{\bf XX Fill in: application to the formal degree ?}

\section{Tori} 
In this paper, we will use the notions of a definable set and  definable function. These refer to being definable in Denef-Pas language. 
Formulas in Denef-Pas language can have variables of three \emph{sorts}: valued field (which will be denoted by $VF$), residue field (denoted by $RF$) and the value group. Even though we will be often be working with ramified extensions, we always start with a local field $F$ with normalized valuation, so the value group is $\Z$ (the $VF$-variables will range over $F$, and so their valuations will be in $\Z$).
Formulas in Denef-Pas language can be interpreted given a choice of a valued field \emph{and a uniformizer of the valuation}. 
We refer the reader to \cite{what's the best ref?} for the definitions of Denef-Pas language, definable sets, etc. 

For us $F$ will always be a non-Archimedean local field, either a finite extension of $\Q_p$, or 
$\F_q\llp t\rrp$, and due to the nature of the notion of a definable set,  all the statements in this paper will hold for all such $F$ of sufficiently large residue characteristic 
(though there will be no effective bound on the characteristic). Since we have to assume that the characteristic is large (relative to some fixed parameters such as the degree of the extension over which our torus splits), we will, in particular, only work with tame tori.  

For a local field $F$, we will denote its ring of integers by $\ri_F$,  its residue field by $k_F$, and let $q_F=\# k_F$. The symbol $\varpi_F$ will always stand for the uniformizer of the valuation on $F$; when there is no possibility of confusion, we might drop the subscript $F$. 
A formula in Denef-Pas language  with $n$ free $VF$-variables, $m$ free $RF$-variables, and $r$ free 
$\Z$-variables 
defines a subset of $F^n\times k_F^m \times \Z^r$. 
We will denote the definable set $F^n\times k_F^m \times \Z^r$ itself by $VF^n\times RF^m\times \Z^r$. In earlier works on motivic integration this set was denoted by $h[n,m,r]$. 


\subsection{Field extensions}
We start by setting up the framework for working with tori, and later, general reductive groups, in Denef-Pas language, following \cite{cluckers-hales-loeser}, \cite{CGH-2}, and \cite{hales:transfert}. 


We encode field extensions in the same way as in \cite{CGH-2}. Namely, we fix a finite group $\Gamma$, and realize all tori over a given local field $F$ that split over a Galois extension $L$ with the Galois group 
isomorphic to $\Gamma$ as members of a family of definable sets, for all $F$ of sufficiently large residue characteristic. 

Thus, once and for all, we fix a pair $(\Gamma,I)$,  with $\Gamma$ an abstract finite group, with fixed enumeration of its elements, $\Gamma=\{\sigma_1, \dots, \sigma_m\}$, and 
$I=\{\sigma_1, \dots, \sigma_e\}$
a normal subgroup (which will play the role of the inertia subgroup). We will also need to use the maximal unramified sub-extension of $L$.
There is an exact sequence of Galois groups, 
$$ 1\to I \to \gal(L/F)\to \gal(L^{ur}/F)\to 1.$$
We 
%fix the ramification index $e$, and make the convention that the elements
%$\sigma_1, \dots, \sigma_e$ of $\Gamma$ constitute the group $\gal(L/L^{\ur})$, and 
make the convention that $\sigma_m$ projects to a generator of $L^{\ur}$ over $F$. 
In order to encode the data of the extension tower $L/L^{\ur}/F$, we let $f=m/e$, and first introduce the parameters $b_0,\dots, b_{f-1}$ ranging over $\ri_F$, which serve as the coefficients of a polynomial whose root generates $L^{\ur}$. 
We identify $L^{ur}$ with   $F[x]/(p(x))$, where $p(x)=x^f+b_{f-1}x^{f-1}+ \dots + b_0$. 
Further, we introduce the parameters $c_0, \dots, c_{e-1}$, which range over 
$L^{\ur}$ (i.e., can be thought of as $f$-tuples of elements of $F$), and which play the role of the coefficients of a polynomial generating the extension $L/L^{\ur}$. 
We impose the conditions (all of which are definable by Denef-Pas language formulas) that: 
\begin{enumerate}
\item the polynomial $p(x)=x^f+b_{f-1}x^{f-1}+ \dots + b_0$ is irreducible over $F$ and its reduction 
$\mod \varpi_F$ is irreducible over $k_F$ and generates a degree $f$ extension of $k_F$. 
This ensures that $F[x]/(p(x))$ is a degree $f$ unramified extension. 
We denote this extension by $L^\ur$, and once and for all fix an identification with $F^f$ as 
an $F$-vector space. 
\item  the elements $c_0, \dots, c_{e-1}$ define an irreducible polynomial over $L^\ur$ with splitting field of degree $e$; this is the field $L$, and we also fix its identification with 
$(L^\ur)^e$ as an $L^\ur$-vector space, with allows us to think of its elements as $m$-tuples of elements of $F$. 
\item $\sigma_1, \dots \sigma_m$ are distinct field automorphisms of $L$ over $F$, and 
 $\sigma_1, \dots \sigma_e$ fix $L^\ur$.
\item The restriction of $\sigma_m$ to $L^\ur$ is a generator of $\gal(L^\ur/F)$. 
\item the group $\{\sigma_1, \dots, \sigma_n\}$ is isomorphic to $\Gamma$.
\end{enumerate}
We denote by $\cE_\Gamma$ the space of parameters $(b_0, \dots, b_{f-1}, c_0, \dots, c_{e-1}, \sigma_1, \dots, \sigma_n)$ with these properties, thought of as a definable subset of some large affine space over $F$. 
Given a field $F$ of sufficiently large residue characteristic, every element of $\cE_\Gamma$ gives rise to a tower of field extensions $L/L^\ur/F$ with $\gal(L/F)$ isomorphic to $\Gamma$ and satisfying all the above conditions. Note, however, that we cannot quite distinguish between 
isomorphism classes of extensions, but all extensions of a given field $F$ with Galois group 
$\Gamma$ arise via this construction.   
\begin{example}
The two ramified quadratic extensions of $F$ appear as members of the same family, one for the 
polynomial $x^2-\varpi$, another for the polynomial $x^2-\epsilon\varpi$, where $\epsilon$  is any non-square unit. We also recall that the interpretation of formulas in Denef-Pas language depends not just on the field, but also on the choice of the uniformizer of the valuation. Thus, in this case, a different choice of the uniformizer would cause these two extensions to switch places, but they would still both appear. 
\end{example}
\begin{rem} We observe that in this context we cannot use standard number-theoretic constructions involving objects such as the cyclotomic polynomial, since we need to be able to let $F$ with its residue characteristic vary, and thus polynomials whose degree depends on $p$ are not allowed. For the same reason, we cannot use the Frobenius automorphism.  
\end{rem}
Our goal is to construct all tori over $F$ that split over a Galois extension with Galois group 
$\Gamma$ and inertia subgroup $I$, and the identity components of their N\'eron models, as 
definable sets fibered over $\cE_\Gamma$.  

\subsection{The set-up}
We follow \cite{cluckers-hales-loeser} in the treatment of the character and cocharacter lattices.
We will think of a split rank $n$ torus ${\bT}^\spl$ as the set of diagonal matrices in $\GL_n$, which is, clearly, definable. Its character lattice $X^\ast({\bT}^\spl)$ is viewed as a subset of the coordinate ring of $\bT^\spl$, whose elements are, explicitly, polynomials in $n^2$ variables. 
We also fix, once and for all, an isomorphism between $X^\ast(\bT^\spl)$ and $\Z^n$. 
\footnote{As pointed out in \cite{cluckers-hales-loeser}, this makes it tempting to represent 
$X^\ast(\bT^\spl)$ by constants (or variables) of the $\Z$-sort in Denef-Pas language, but this, indeed, is a wrong way to proceed.}
From now on, together with the groups $\Gamma$ and $I$, we shall fix a homomorphism
$\theta:\Gamma \to \GL_n(\Z)$, as part of the initial completely field-independent `fixed data' built into all our constructions.
Moreover, the set $H^1(\Gamma, \GL_n(\Z))$ is completely field-independent, and therefore can also be fixed from the start. The trivial class is represented by the map $\theta$.  We can, and will, fix a set of representatives of these cohomology classes, and denote them by $Z=\{z_1, \dots, z_m\}$. We will use the $z_i$ in Denef-Pas formulas, thinking of them as $|\Gamma|$-tuples of $n\times n$-matrices whose integer entries are \emph{constants of the valued  field sort}. 
 
For $\epsilon=(\bar b, \bar c, \sigma_1, \dots, \sigma_n)\in \cE_\Gamma$, we get an extension tower 
$L_\epsilon/L_\epsilon^\ur/F$. For $z\in Z$, define the action of $\gal(L/F)$ on $\bT^\spl(L)$ by
$\sigma\cdot  t:=z(\sigma)\sigma^{-1}t$, where $t\in \bT^\spl(L)$, and for $\sigma\in \Gamma$, 
$\sigma^{-1}t$ is the coordinate-wise action of $\sigma^{-1}$, which is thought explicitly as an $n\times n$-matrix with entries that are definable functions of the parameters $(\bar b, \bar c)$, on the coordinates of $t$, which are thought of as elements of $L$, i.e., as tuples of $F$-variables. 

The $F$-tori that split over $L$ are in bijection with  the set $Z$; 
for $z\in Z$, we denote the corresponding algebraic $F$-torus by $\bT_z$. 
By the above construction, the set of its $F$-points $\bT_z(F)$ is definable by a formula in Denef-Pas language (using parameters in $\cE_\Gamma)$, since it is the set of fixed points of the action defined in the above paragraph. 

\subsection{The connected component of the N\'eron model: induced case}
Let us first consider the case $\bT=\Res_{L/F} \bT^\spl$, where $L$ is a finite Galois extension with Galois group $\Gamma$ as above. This is equivalent to choosing a cocycle $z$ that gives the action on $X^\ast(\bT^\spl)\simeq \Z^n$ induced from the trivial action. 
% Note that this condition on $z$ is definable. ({\bf XX Do we need this? If no, maybe remove, if yes, provide detail}).
Write $\Ner{T}$ for the N\'eron model of $\bT$, and $\NerC{T}$ for its identity component.  Then $\Ner{T}(\ri_F) = \bT(F) = L^\times$ consists of tuples of $F$-variables that are not all zero, and $\NerC{T}(\ri_F) = \ri_F^\times$ consists of tuples of $F$-variables all with non-negative valuation -- both are definable conditions.

We observe that in this case the component group of $\Ner{T}$ can be identified with $\Z^n$, and here we interpret $\Z^n$ as the value sort. The map $v:\Ner{T} \to \pi_0(\Ner{T})$ is the component-wise valuation: we treat an element of $L$ as an $n$-tuple of elements of $F$, and $v$ maps it to the  tuple of valuations of the components. 

\subsection{The connected component of the N\'eron model: an outline}

We want to show that the identity component of the N\'eron model is definable.  Consider the short exact sequence of $\Gal(L/F)$-modules
\[
0 \to X^*(\bT) \to \Ind_{G_F}^{G_L} X^*(\bT) \to X' \to 0
\]
defined in \cite{brahm}*{Satz 0.4.4} and the dual sequence of tori
\[
1 \to \Res_{L/F}^1(\bT_L) \to \Res_{L/F} (\bT_L) \to \bT \to 1.
\]
Write $\bS = \Res_{L/F} (\bT_L)$ and $\Ner{S}$ for the N\'eron model of $\bS$; $\bR = \Res_{L/F}^1 (\bT_L)$ and $\Ner{R}$ for its N\'eron model. 
%We have the exact sequence of N\'eron models 
%\[
%1 \to \Ner{R} \to \Ner{S} \to \Ner{T} \to 1, 
%\]
%and the resulting sequence of component groups 
We seek a formula that tests when a given element of $\bT(F)$ actually lies in $\NerC{T}(\ri_F) \subseteq \Ner{T}(\ri_F) = \bT(F)$. The set $\bS(F) \cong (L^\times)^n$ is definable, as is the valuation map to the component group of its N\'eron model, $v : \bS(F) \to \pi_0(\Ner{S}) \cong \Z^n$.  The image of $\pi_0(\Ner{R})$ in $\pi_0(\Ner{S})$ is independent of $F$ (as can be seen from \cite{bertrapelle-gonzales:13b}*{Prop. 3.8}), so we may fix generators $r_1, \ldots, r_m$ for use in our formulas.  The functor $\bT \mapsto \pi_0(\Ner{T})$ is right exact \cite{bertrapelle-gonzales:13b}*{Prop. 3.8}, so to determine whether a given element $t \in \bT(F)$ lies in $\NerC{T}(\ri_F)$, it suffices to check if there exists an element of $\bS(F)$ projecting to $t$ with valuation in the span of $r_1, \ldots, r_m$.  This is a definable condition.
Thus, to implement this construction precisely, we need to introduce the fixed choice 
$r_1, \dots, r_m$ and the suitable compatibility conditions. 


\subsection{}More precisely, we need to make fixed choices of $z_1, \dots z_n$. 
There is a compatibility condition -- the quotient of $\Z^n$ by the lattice generated by this fixed choice should be $(X_\ast)_I$, which is already fixed at this point. 
Rename the $z$'s -- $z$ is reserved for cocycle.  

\section{General reductive groups}
{\bf XX Here want to prove that $G_{x, 0}$ is definable; canonical measure is motivic; formal degree is motivic. }
\begin{prop} Let $G$ be a tamely ramified quasi-split connected reductive group defined over a local field $F$. 
Let $x$ be a special point in the building of $G$. 
Then the canonical compact subgroup $G_x=G_{x, 0}$ associated with $x$ via Bruhat-Tits theory is definable. 
\end{prop}
\begin{proof} Let $T$ be a maximal split torus in $G$.
By definition, $G_x$ is generated by $T_x$ and $U_\psi$, where 
 
Now that we know that $T_x$ is definable, the proof of identical to that of Lemma 3.4 in \cite{CGH-2}.  
\end{proof} 


\end{document}





























\section{What is a motivic (exponential) function?}\label{sec:motfun}
Informally, motivic functions are built from definable functions in the Denef-Pas language and motivic \emph{exponential} functions are built from motivic functions, additive characters on local fields, and definable functions serving as arguments of the characters. As such, they are given independently of the local field and can be interpreted in any local field and  with any additive character. 
Let us recall the definition of the Denef-Pas language first, and proceed to motivic (exponential) functions afterwards, where we slightly simplify the terminology and notation of \cite{CGH-1}.

\subsection{Denef-Pas language}\label{sub:DP}
The Denef-Pas language is a first order language of logic designed for working
with valued fields. Formulas in this language will allow us to uniformly {handle} sets and functions for all {local} fields.
We start by defining two sublanguages of the language of
Denef-Pas: the language of rings and the Presburger language.
\subsubsection{The language of rings}

Apart from the symbols for variables $x_1, \dots, x_n,\dots$ and the
usual logical symbols equality `$=$', parentheses `$($', `$)$', the quantifiers `$\exists$', `$\forall$', and the logical operations conjunction `$\wedge$', negation `$\neg$', disjunction `$\vee$', the language of rings consists of the following symbols:
\begin{itemize}
\item constants `$0$', `$1$';
\item binary functions `$\times$', `$+$'.
\end{itemize}

A (first-order) formula in the language of rings is any syntactically correct formula
built out of these symbols. One usually omits the words `first order'.

If a formula in the language of rings has $n$ free (i.e.\ unquantified) variables then it defines a subset of $R^n$ for any ring $R$.
For example the formula ``$\exists x_2\, (x_2\times x_1 = 1)$'' defines the set of units $R^\times$ in any ring $R$; note that
quantifiers {(by convention)} always run over the ring in question.
Note also that quantifier-free formulas in the language of rings define constructible sets (as they appeared in classical algebraic geometry).

\subsubsection{Presburger language}\label{Pres}
A formula in Presburger's language is built out of variables running over $\Z$, the logical symbols (as above) and
symbols `$+$', `$\le$', `$0$', `$1$', and for each $d=2,3,4,\dots$, a symbol `$\equiv_d$' to denote the binary
relation $x\equiv y \pmod{d}$.
Note the absence of the symbol for multiplication.

Since multiplication is not allowed, sets and functions defined by formulas in the Presburger language are in fact very basic, cf.~\cite{CPres} or \cite{Presburger}. By \cite{Presburger}, quantifiers are never needed to describe Presburger sets, so
all definable sets are of simple form. Thus, the subsets of the line $\ZZ$ defined by Presburger formulas are finite unions of arithmetic progressions (in positive or negative direction) and points. This simple observation is at the source of the simple forms of the bounds in Theorems \ref{thm:presburger-fam}, 
\ref{thm:transfer-fam},  and \ref{thm:main2}, \ref{thm:main}.
 





\subsubsection{Denef-Pas language}\label{DP}
The Denef-Pas language is a three-sorted language in the sense that its formulas speak about
three different ``sorts'' of elements: those of the valued field, of the residue field and of the value group (which will always be $\Z$ in our setting).
Each variable in such a formula only runs over the elements of one of the sorts,
i.e., there are three disjoint sets of symbols for the variables of the different sorts.
For a formula to be syntactically correct, one has to pay attention to the sorts when composing functions
and plugging them into relations.



In addition to the variables and the logical symbols, the formulas use the following symbols.
\begin{itemize}
\item In the valued field sort:
the language of rings.
\item In the residue field sort: the language of rings.
\item In the $\Z$-sort: the Presburger language.
\item a symbol $\ord(\cdot)$ for the valuation map from the nonzero elements of the valued field sort to the $\Z$-sort, and {a symbol} $\ac(\cdot)$ for the so-called angular component, which is a {multiplicative} function from the valued field sort to the residue field sort (more about this function below).
\end{itemize}

A formula in this language can be interpreted in any discretely valued field $F$ which comes with a
uniformizing element $\varpi$,
by letting the variables range over $F$, {over its} residue field $k_F$, and {over} $\Z$,
respectively, depending on the sort they belong to;
$\ord$ is the valuation map (defined on $F^\times$ and such that $\ord(\varpi)=1$), and $\ac$ is defined as follows.
If $x$ is a unit (that is, $\ord(x)=0$), then $\ac(x)$ is the residue of $x$ modulo $\varpi$ (thus, an element of the residue field); for all other
nonzero $x$, one puts $\ac(x) :=  \varpi^{-\ord(x)}x \bmod (\varpi)$.  Thus, for $x\neq 0$, $\ac(x)$ is the residue class of the 
first non-zero coefficient of the $\varpi$-adic expansion of $x$. Finally, we define $\ac(0)=0$.

Thus, a formula $\varphi$ in this language with $n$ free valued-field variables, $m$ free residue-field variables, and $r$ free $\Z$-variables gives naturally, for each discretely valued field $F$, a subset $\varphi(F)$ of
$F^n\times \rf_F^m\times \Z^r$: namely, $\varphi(F)$ is the set of all the tuples where the interpretation of $\varphi$ in $F$ is ``true'' (where a variable appearing inside a quantifier runs over either $F$, $k_F$, or $\ZZ$, respectively, depending on its sort).


\subsubsection{Adding constants to the language}
\label{sect:ldpo}

In many situations one needs to work with geometric objects defined over some fixed base number field, or over its ring of integers $\ri$. In such situations,
on top of the symbols for the constants that are already present in the above language (like $0$ and $1$), we will add to the Denef-Pas language all elements of
$\ri[[t]]$ as extra symbols for constants in the valued
field sort.
We denote the resulting language by $\ldpo$.

Given a complete discretely valued field $F$    {which} is an algebra over $\ri$ via {a chosen}
algebra homomorphism $\iota:\ri\to F$,    {with} a chosen uniformizer $\varpi$ of the valuation ring $\cO_F$ of $F$, one
can interpret the formulas in $\ldpo$ as described in the previous subsection, where
the new constants from $\ri$ are interpreted as elements of $F$ by using $\iota$, the constant symbol $t$ is interpreted as the uniformizer $\varpi$, and thus, by the completeness of $F$, elements of $\ri[[t]]$ can be naturally interpreted in $F$ as well.

\begin{defn}\label{AO}
Let $\cC_\ri$ be the collection of all triples $(F, \iota , \varpi)$, where $F$ is a non-Archimedean local
field which allows at least one ring homomorphism from $\ri$ to $F$,  the map $\iota:\ri\to F$ is such a ring homomorphism, and $\varpi$ is a uniformizer for $F$.
Let $\cA_\ri$ be the collection of triples $(F, \iota , \varpi)$ in $\cC_\ri$
with
$F$ of characteristic zero, and let $\cB_\ri$ be the collection of the triples $(F, {\iota} , \varpi)$ such that $F$ has positive characteristic.
\green{Here by a non-Archimedean local field we mean a finite extension of $\Q_p$ or $\F_p\llp t\rrp$}.  

Given an integer $M$, let $\cC_{\ri, M}$ be the collection of $(F,{\iota},\varpi)$ in $\cC_\ri$ such that the residue field of $F$ has characteristic larger than $M$, and similarly for $\cA_{\ri, M}$
and $\cB_{\ri, M}$.
\end{defn}

%\red{I am not sure what the referee in report (a) means; I think they think that this definition defines non-arch. local fields. Can you look at the 
%first comment in report (a) and see how we can address it? BTW, the reference to 2.4.1 in that comment should be 2.1.4).} 
 
Since our results and proofs are independent of the choices of the map $\iota$ and the
uniformizer $\varpi$ (\green{in the sense that we ultimately prove all our statements for all fields $F\in \cC_{\ri}$ with sufficiently large residue characteristic, with any choices of $\iota$ and $\varpi$}), we will often just write $F\in \cC_\ri$, instead of naming the whole triple.
For any $F\in \cC_{\ri}$, write $\cO_F$ for the valuation ring of $F$, $k_F$ for its  residue field, and $q_F$ for the cardinality of $k_F$.
%\red{Please see if you can elaborate on this (see report (b)).}

\subsection{Definable sets and motivic  functions}\label{subsub:functions}

The $\ldpo$-formulas introduced in the previous section allow us to obtain a field-independent notion of subsets of $F^n\times k_F^m\times \ZZ^r$ for $F\in \cC_\ri$.

\begin{defn}\label{defset}
A collection
$X = (X_F)_{F}$ of subsets $X_F\subset F^n\times \rf_F^m\times \Z^r$ for $F$ in $\cC_\ri$ is called a \emph{ definable set} if there is an
$\ldpo$-formula $\varphi$ \green{and an integer $M$} such that $X_F = \varphi(F)$ for each $F$ \green{with residue characteristic at least $M$ (cf. Remark \ref{rem:largep})}, where $\varphi(F)$ is as explained at the end of \S \ref{DP}.
\end{defn}
%\red{Can you address the second comment of report (a) here? I think I almost agree with the referee, but I know you meant to say for all $F$. Can you add a sentence or two explaining this? In fact, I put in the blue remark just a bit below, also addressing his later comment that was asking to move some sentences from p.5. Maybe you can just make that remark precise and that would fix this issue?} 

By Definition \ref{defset}, a ``definable set'' is actually a collection of sets indexed by $F\in \cC_\ri$; such practice is not uncommon in model theory and has its analogues in classical algebraic geometry.
A particularly simple definable set is $(F^n\times k_F^m\times \ZZ^r)_F$ with $\cO=\Z$, for which  we introduce
the simplified notation
${\rm VF}^n\times {\rm RF}^m\times \ZZ^r$, where ${\rm{VF}}$ stands for valued field and ${\rm{RF}}$ for residue field.
We apply the typical set-theoretical notation to definable sets $X, Y$,    e.g.,
$X \subset Y$ (if $X_F \subset Y_F$ for each $F \in \cC_\ri$), $X \times Y$, and so on.

\begin{defn}\label{deffunct}
For definable sets $X$ and $Y$, a collection $f = (f_F)_F$ of functions $f_F:X_F\to Y_F$ for $F\in\cC_\ri$ is called a definable function and denoted by $f:X\to Y$ if
the collection of graphs of the $f_F$ is a definable set.
\end{defn}

\begin{rem}\label{rem:largep}
There is a subtle issue here caused by the fact that the same definable set can de defined by different formulas. Technically, \green{it would be more elegant to think of a definable set} as an equivalence class of what we have called definable sets in the Definition \ref{defset} above, where we call two such definable sets equivalent if they are the same for all $F$ with sufficiently large residue characteristic.
In order not to complicate notation, we will not emphasize this point, but because of this 
all the results presented in this article will only be valid for sufficiently big residue characteristic.
In particular, for any of the uniform objects $X$ we introduced in Defitions~\ref{defset}, \ref{motfun}, \ref{expfun},
we are actually only interested in $(X_F)_{F \in \cC_{\ri,M}}$ for $M$ sufficiently big.
\end{rem}


There are precise quantifier elimination statements for the Denef-Pas language,
which  lead to the study of the geometry of definable sets and functions for $F$ in $\cC_{\ri,M}$, where often $M$ is large and depends on the data defining the set. To give an overview of these results is beyond of the scope of  this survey, but the reader may consult \cite{CCL:metric}, \cite{CCL:lipschitz}, \cite{CH: Lipschitz} and others.




We now come to motivic  functions, for which definable functions are the building blocks as mentioned above.
{We note that while definable functions, by definition, have to be ${\rm VF}^n\times {\rm RF}^m\times \ZZ^r$-valued for some $m,n,r$, the 
\emph{motivic} functions are built from \green{definable sets and  functions}, and can be thought of as complex-valued functions (though \green{values outside of $\Q$ appear only in Definition \ref{expfun}}).  This does not require thinking of rational or complex numbers in any logic context; instead, these are just usual complex-valued functions that happen to be built from definable ingredients in the way prescribed by the following definitions.
 
\begin{defn}\label{motfun}
Let $X = (X_F)_F$ be a definable set.
A collection $f = (f_F)_F$ of functions $f_F:X_F\to\CC$ is called \emph{a motivic function} on $X$ if and only if
there exist integers
$N$, $N'$, and $N''$, such that, for all $F\in \cC_\ri$, 
$$
f_F(x)=\sum_{i=1}^N   q_F^{\alpha_{iF}(x)} ( \# (Y_{iF})_x  )  \big( \prod_{j=1}^{N'} \beta_{ijF}(x) \big) \big( \prod_{\ell=1}^{N''} \frac{1}{1-q_F^{a_{i\ell}}} \big), \mbox{ for } x\in X_F,
$$
for some
\begin{itemize}
\item nonzero integers $a_{i\ell}$, 
\item definable functions $\alpha_{i}:X\to \ZZ$ and $\beta_{ij}:X\to \ZZ$, \item definable sets  $Y_i\subset X\times {\rm RF}^{r_i}$,
\end{itemize}
where, for $x\in X_F$,  $(Y_{iF})_x$ is the finite set $\{y\in k_F^{r_i}\mid (x,y)\in Y_{iF}\}$.
\end{defn}

\subsection{Motivic exponential functions}\label{subsub:expfunctions}

For any local field $F$, let $\scD_F$ be the set of the additive characters on $F$ that are trivial on the maximal ideal $\mathfrak m_F$ of $\cO_F$ 
and nontrivial on $\cO_F$.
For $\Lambda \in \scD_F$,
we will write $\bar\Lambda$ for the induced additive character on $k_F$.
{Note that additive characters of $\Q_p$, for example, are obtained from characters of the finite field $\mathbb F_p$, which are exponential functions. 
Thus, expressions involving additive characters of $p$-adic fields often give rise to exponential sums, and this explains the term ``exponential'' in the definition below.} 

\begin{defn}\label{expfun}
Let $X  = (X_F)_F$ be a definable set.
A collection $f = (f_{F,\Lambda})_{F,\Lambda}$ of functions $f_{F,\Lambda}:X_F\to\CC$ for $F\in \cC_\ri$ and $\Lambda\in \scD_F$ in is called \emph{a motivic exponential function} on $X$ if 
there exist integers $N>0$ and $r_i\geq 0$, motivic functions $f_i=(f_{iF})$, definable sets $Y_i\subset X\times {\rm RF}^{r_i}$ and definable functions $g_i:Y_i\to {\rm VF}$ and $e_i:{Y_i}\to {\rm RF}$ for $i=1,\ldots,N$, such that for all $F\in \cC_\ri$ and all $\Lambda\in \scD_\K$ %, $f_F$ has the form
$$
f_{F,\Lambda}(x)=\sum_{i=1}^N   f_{iF}(x)\Big( \sum_{y \in (Y_{iF})_x}\Lambda\big(g_i(x,y)\big) \cdot \bar \Lambda\big(e_i(x,y)\big)\Big)\mbox{ for all } x\in X_F.
$$
\end{defn}

Compared to Definition \ref{motfun}, the counting operation $\#$ has been replaced by taking exponential sums, which makes the motivic exponential functions a richer class than the motivic functions.
{Indeed, note that the sum as above gives just $\#(Y_{iF})_x$ if $g_i=0$ and $e_i=0$.}
Motivic (exponential) functions form classes of functions which are stable under integrating some of the variables out, see Theorem \ref{thm:mot.int.} below. In fact, the search for this very property led {R. Cluckers and F. Loeser to this definition for motivic exponential functions, introduced in \cite{cluckers-loeser:fourier}}.

\subsection{Families}
%   {There are natural notions of ``uniform families'' of the objects introduced in Definitions~\ref{defset},
%\ref{motfun} and \ref{expfun}. We fix the following terminology.}
\green{It is often convenient to talk about families of objects; in particular}, we will talk about families of definable sets and of definable, motivic, or motivic exponential functions, even though formally these notions are just special cases of the above Definitions ~\ref{defset},
\ref{motfun} and \ref{expfun}.

\begin{defn}\label{deffam}
Suppose that $A$ is a definable set.
A {\emph{family of definable sets}} with parameter in $A$ is a definable set $X\subset A\times Y$ with some definable set $Y$.
We denote such a family by $(X_a)_{a \in A}$.
For each $F\in \cC_\ri$ and $a\in A_F$, the sets $X_{F,a}:= \{y\in Y_F\mid (a,y)\in X_F  \}$
are called the family members of the family $X_F = (X_{F,a})_{a\in A_F}$.

Similarly, for $(X_a)_{a \in A}$ as above, an (exponential) motivic, resp.~definable, function $f$ on $X$
can be considered as a \emph{{family $(f_a)_{a \in A}$} of (exponential) motivic}, resp.~\emph{definable
functions} with parameter in $A$. Its members are functions $f_{F,a}$ or $f_{F,\Lambda,a}$ for $F\in \cC_\ri$, $\Lambda\in \scD_F$ and $a\in A_F$.
\end{defn}

Whenever we call a specific function $f_{F}$ motivic, we implicitly think of $F$ as varying 
and we mean that there exists an $M > 0$ and a
motivic function    {$g$ with $f_F = g_F$} for all $F \in C_{\cO, M}$.
Likewise, we think of $F$ as varying whenever a specific set $X_{F}$ or function $f_{F}$ 
is called definable.


\subsection{Measure and integration}

To integrate a motivic function $f$ on a definable set $X$, we need a uniformly given family of measures on each $X_F$. For $X = {\rm VF}$, we put the Haar measure on $X_F = F$ so that $\cO_F$ has measure $1$; on
$k_F$ and on $\ZZ$, we use the counting measure. To obtain measures on
arbitrary definable sets $X$, we use ``definable volume forms'', as follows.

Any $F$-analytic subvariety of $F^n$, say, everywhere of equal dimension, together with an $F$-analytic volume form, carries a natural measure associated to the volume form, cf.~\cite{Bour} and \cite{weil:adeles}*{\S 2.2}.
\red{Raf, could you please make the Bourbaki reference precise?} 
\blue{This carries over to the
motivic setting,  namely, there 
is a notion of a definable volume form on a definable set, introduced in \cite{cluckers-loeser}*{\S 8}.}
In a nutshell, a definable volume form $\omega$ on a definable set $X$ is a volume form defined by means of definable functions on definable coordinate charts, see \cite{CGH-2}*{\S 3.5.1} and references therein for details.
% see \cite{cluckers-loeser-nicaise}.
Such a definable volume form specializes to a volume form $\omega_F$ on $X_F$ for every $F$
with sufficiently large residue characteristic; more precisely, a definable volume form specializes to a top degree differential form for every $F$, but its non-vanishing can be assured only when the residue characteristic of $F$ is sufficiently large.
 By a
``motivic measure'', we mean a family of measures on each $X_F$ that arises in this way from a definable
volume form.

%Any algebraic volume form defined over $\ri$ on a variety $V$ over $\ri$ gives rise to a definable
%volume form on $V(F)$ for    {$F\in \cC_{\cO, M}$ for a sufficiently large constant $M$}, in the above sense.
%However, often one deals with volume forms on, say, orbits of elements of a group defined over a local field, or other volume forms defined over a local field, for which it is not always easy to obtain that they can be defined in a way which does not depend on the particular local field.



The classes of motivic (exponential) functions are natural to work with for the purposes of integration, as shown by the following theorem, which generalizes a similar result of
\cite{cluckers-loeser:fourier} to the class of functions summable in the classical sense of $L^1$-integrability.


\begin{thm}\cite{CGH-1}*{Theorem 4.3.1}\label{thm:mot.int.}
Let $f$ be a motivic  function, resp.~a motivic exponential function, on $X\times Y$ for some definable sets $X$ and $Y$, with $Y$ equipped with a motivic measure $\mu$. Then there exist a motivic  function, resp.~a motivic exponential function, $\green{I}$ on $X$ and an integer $M>0$ such that for each $F\in \cC_{\ri,M}$, each $\Lambda\in \scD_F$
and for each $x\in X_F$ one has
$$
\green{I}_F(x) = \int_{y\in Y_F}f_F(x,y)\,d\mu_F, \mbox{ resp. } I_{F,\Lambda}(x) = \int_{y\in Y_F}f_{F,\Lambda}(x,y)\,d\mu_F,
$$
whenever the function $Y_F\to\CC:y\mapsto f_F(x,y)$, resp.~$y\mapsto f_{F,\Lambda}(x,y)$ lies in $L^1(Y_F, \mu_F)$.
\end{thm}
%\red{Note: I changed the notation $g$ to $I$ here in order to address a comment from report (b) about the confusion that happens later when we interpolate a function by an integrable function. Hope it's OK.}


\section{New transfer principles for motivic exponential functions}\label{sec:trpr}
Theorem \ref{thm:mot.int.} leads to the transfer of integrability and transfer of boundedness
principles, which we quote from \cite{CGH-1}.
(For simplicity, we quote the version without parameters; that is, below we present 
\cite{CGH-1}*{Theorem 4.4.1} and
\cite{CGH-1}*{Theorem 4.4.2} in the case that the parameter space is a point).

\begin{thm}\cite{CGH-1}*{Theorem 4.4.1}\label{thm:ti}
Let $f$ be a motivic  exponential
function on ${\rm VF}^n$.
Then there exists $M>0$, such that for the
fields $F\in\cC_{\ri, M}$,
the truth of the statement that $f_{F,\Lambda}$
is (locally) integrable
for all $\Lambda \in \scD_F$
depends only on the isomorphism class of the residue field of $F$.
\end{thm}

\begin{thm}\cite{CGH-1}*{Theorem 4.4.2}\label{thm:bounded}
Let $f$ be a motivic exponential
function on ${\rm VF}^n$.
Then, for some $M>0$, for the fields $F\in\cC_{\ri, M}$,
the truth of the statement that $f_{F, \Lambda}$ is (locally) bounded
for all $\Lambda \in \scD_F$
depends only on the isomorphism class of the residue field of $F$.
\end{thm}

In the versions with parameters of these results, $f$ is a family of motivic exponential functions
with parameter in a definable set $A$. In that case, each of the two theorems states that there exists a
motivic exponential function \green{$h$} on $A$ such that the  statement under consideration is true for
$f_{F,\Lambda,a}$ iff $h_{F,\Lambda}(a) = 0$, that is, we relate the \emph{locus of integrability}, resp. the \emph{locus of boundedness} of the motivic (exponential) function $f$ to the zero locus of another motivic (exponential) function \green{$h$}.

\section{New ``uniform boundedness'' results for motivic functions}\label{sec:ub}

It turns out that the field-independent description of motivic functions leads not only to
transfer principles, but also to uniform (in the residue characteristic of the field) bounds for those motivic functions whose specializations are known to be bounded for individual local fields (for now, for motivic functions without the exponential).  We quote the following theorems from \cite{S-T}.

\begin{thm}\label{thm:presburger-fam}\cite{S-T}*{Theorem B.6}
Let $f$ be a motivic  function on $W \times \Z^n$,
where $W$ is a definable set. 
If for almost all fields $F\in \cA_{\ri}$ or almost all fields $F\in \cB_{\ri}$ there exists any (set-theoretical) function 
$\alpha^F:\ZZ^n\to\R$ (depending on $F$) such that
$$
| f_F(w,\lambda) |_{\C} \le \alpha^F(\lambda) \mbox{ on } W_F \times \Z^n,
$$
then 
\green{
there exist integers $a,b$ and $M$ such that}
for all $F\in \cC_{\ri, M}$, 
$$
| f_F(w,\lambda) |_{\C} \le q_F^{a+b \|\lambda\|} \mbox{ on } W_F \times \Z^n,
$$
where    {$\|\lambda\| = \sum_{i=1}^n |\lambda_i|$}, and where \green{$|\cdot|_\C$ is the norm on $\C$}.
\end{thm}

%\red{Note: changed $||_\R$ to $||_\C$  above in response to report (a); it doesn't matter here since there's not exponentials, but agrees better with the definition for motivic functions  we have. Also, made a change coming from Tasho's comment about field-independence. Please re-read this theorem.}

We observe that in the case with $n=0$,
the theorem yields that if a motivic  function $f$ on $W$ is such that $f_F$ is bounded on $W_F$, {say,  for each $F\in \cA_{\ri, M}$}, then the bound for $|f_F|_\C$  can be taken to be $q_F^a$ for some $a\geq 0$, uniformly in $F$ with large residue characteristic, 
and moreover, \green{the same constant works for $F\in \cB_{\ri, M}$ as well}.
%\red{Note: I made the statement stronger, but I think this is what we actually prove, and what is needed in the appenddix. This is very important: this theorem in the appendix will need to be fixed along these lines, since as it was stated it did not make sense, and also boundedness a priori is only known in characteristic zero. So please check if you agree with this statement.}

The following statement allows one to transfer bounds which are known for local fields of
characteristic zero to local fields of positive characteristic, and vice versa.

\begin{thm}\label{thm:transfer-fam}\cite{S-T}*{Theorem B.7}
Let $f$ be a  motivic function on $W \times \Z^n$,
where $W$ is a definable set, and let $a$ and $b$ be integers. Then there exists $M$ such that, for any $F\in \cC_{\ri, M}$,
whether the statement
\begin{equation}\label{transfer}
f_F(w,\lambda)\le q_F^{a+b \|\lambda\|} \mbox{ for all } (w,\lambda) \in W_F \times \Z^n
\end{equation}
holds or not, only depends on the isomorphism class of the residue field of $F$.
\end{thm}

\subsection{A few words about proofs.}


The common strategy in the proofs of Theorems~   {\ref{thm:mot.int.}} -- \ref{thm:transfer-fam} (assuming the set-up of motivic integration developed in \cite{cluckers-loeser} and 
\cite{cluckers-loeser:fourier}) is the following.
The first step is to reduce the problem from
arbitrary motivic exponential functions $f$ to motivic exponential functions $g_i$ of a simpler form,
by writing $f$ as a sum $\sum_i g_i$. For some of the results
(e.g. to obtain a precise description of the locus of integrability or boundedness of $f$),
one needs to prove that not too much cancellation can occur in this sum; this is in fact
one of the main difficulties in the proofs of Theorems~\ref{thm:ti} and \ref{thm:bounded}.

In the next step, we eliminate all the valued-field variables, possibly at the cost of introducing more residue-field and $\ZZ$-valued variables, which uses the Cell Decomposition Theorem.
\red{Could you please put in precise references here and everywhere else in this paragraph (see report (a))?}
Once we have a motivic function that depends only on the residue-field and value-group variables,
we again break it into a sum of simpler terms, and then get rid of the residue-field variables.
Finally, the question is reduced to the study of functions $h : \ZZ^r \to \R$, which are,
up to a finite partition of $\ZZ^r$ into sets    {which are} definable in
the Presburger language (see Section~\ref{Pres}),
sums of products of linear functions in $x \in \ZZ^r$ and of powers of $q_F$, where the power also depends linearly on $x$. For most $x$, the behaviour of such a sum $h$ is governed by a dominant term,
and for a single term, each of the theorems is easy to prove.
To control the locus where no single term of $h$ is dominant, we use the powerful result by Wilkie about
``o-minimality of $\R_{\mathrm{exp}}$''.






\section{Which objects arising in harmonic analysis are definable?}\label{sec:definability}
In the remainder of the article,
when we say ``definable'', we mean, definable in
the Denef-Pas language $   {\ldpo[\Z]}$
(introduced in Section~\ref{sect:ldpo}). 
If one wishes to think specifically of reductive groups defined over a given global field $\bF$ with the ring of integers $\ri$ and its completions, then one can use the language $\ldpo$
instead.
We write $\cA_M$, $\cB_M$, and $\cC_M$ for $\cA_{\Z, M}$, $\cB_{\Z, M}$ and $\cC_{\Z, M}$, respectively (see Definition~\ref{AO}).

\subsection{Connected reductive groups}
We recall that a split connected reductive group over a local field is determined by its
root datum $\Psi=(X^\ast, \Phi, X_\ast, \Phi^\vee)$, see e.g. \cite{springer:lag}*{\S 10}. By an \emph{absolute} root datum of $\bG$ we will mean
the root datum of $\bG$ over the separable closure of the given local field; it is a quadruple of this form.

By an unramified root datum we mean a root datum of an unramified reductive group over a local field $F$, i.e. a quintuple $\xi=(X^\ast, \Phi, X_\ast, \Phi^\vee, \theta)$, where $\theta$
is the action of the Frobenius element of $F^\ur/F$ on the first four components of $\xi$, \blue{compatible with $\Phi\subset X^\ast$, $\Phi^\vee \subset X_{\ast}$, and 
the pairing between $X^\ast$ and $X_\ast$}.
\blue{We will usual denote a quadruple $(X^\ast, \Phi, X_\ast, \Phi^\vee)$ by $\Psi$, and a quintuple that includes Galois action by $\xi$.}

Given an unramified root datum $\xi$,
it is shown in \cite{cluckers-hales-loeser}*{\S 4} that a connected reductive group
$\bG(F)$ with root datum $\xi$ is ``definable using parameters'', for all local 
fields $F\in \cC_M$, where $M$ is determined by the root datum. This result is extended to all connected reductive groups (not necessarily unramified) in \cite{CGH-2}*{\S 3.1}, \cite{S-T}*{B.4.3}.
More precisely, by ``definable using parameters'', we mean that $\bG(F)$ is a member of a family
(in the sense of Definition~\ref{deffam}) of definable connected reductive groups
with parameter in a suitable definable set $Z$. This parameter encodes information about the extension over which $\bG$ splits (see \cite{CGH-2}*{\S 3.1} for details). In the remainder of this article, every definable or motivic object in $\bG(F)$ is parametrized by $Z$, and all results are uniform in the family.


\subsection{Haar measure and Moy-Prasad filtrations} If $\bG$ is unramified, the canonical measure on $\bG(F)$ defined by Gross \cite{gross:motive} is motivic for $F\in \cC_M$ (with $M$ depending only on the absolute root datum of $\bG$), see \cite{cluckers-hales-loeser}*{\S7.1}.
For a root datum defining a ramified group $\bG$,    {to the best of our knowledge, this is presently not known}.    {However, in \cite{S-T}*{Appendix B} we prove} that
there is a constant $c$ that depends only on absolute the root datum, such that Gross' measure can be renormalized by a factor between $q^{-c}$ and $q^c$ (possibly depending on $F$) to give a motivic measure on $\bG(F)$,
where $q=p^r$ is the cardinality of the residue field of $F$, and
$p$ is assumed large enough so that the group is tamely ramified.

In fact, this question is closely related to whether Moy-Prasad filtration subgroups
$\bG(F)_{x,r}$ \blue{(defined in \cite{moy-prasad:k-types}*{\S 2})} are definable (here $x$ is a point in the building of $\bG(F)$, and $r\ge 0$). This question has been answered in a number of important special cases. Namely,
if $\bG$ is unramified over $F$, and $x$ is a hyper-special point, then
we prove in \cite{CGH-2}*{\S 3} that $\bG(F)_{x, r}$ is definable for all $r\ge 0$.
The same proof (which relies on the split case and taking Galois-fixed points) works in the case when $\bG$ is ramified over $F$, $x$ is a special point,
and $r>0$, but fails for $r=0$ since Galois descent no longer works. This is exactly the reason that the question whether the canonical measure is motivic is open in the ramified case.
We also prove in \cite{CGH-2}*{\S3} that for $x$ an optimal point in the alcove (in the sense of \blue{\cite{moy-prasad:k-types}*{\S 6.1}}) 
%the definition by Adler and DeBacker, \cite{adler-debacker:bt-lie}),
$\bG(F)_{x,r}$ is definable for $r>0$, and for all $r\ge 0$ if $\bG$ is unramified over
$F$, and therefore the $\bG(F)$-domain
$\bG(F)_r$ is definable. We also prove the corresponding results for the Lie algebra.
   {We note that the first results on definability of $\bG(F)_{x,0}$ in some special cases appeared in \cite{diwadkar:thesis}.}


\subsection{Orbital integrals and their Fourier transforms}\label{sub:oi}
There are two ways to think of an orbital integral: as a distribution on the space $C_c^\infty(\bG(F))$ of locally constant compactly supported functions on $\bG(F)$ (or, respectively, on the Lie algebra), or to fix a test function $f$ on $\bG(F)$ (respectively, on $\fg(F)$), and consider the orbital integral
$\cO_\gamma(f)$ (respectively, $\cO_X(f)$) as a function on $\bG(F)$ (respectively, on the Lie algebra $\fg(F)$).

The earliest results concerning definability of orbital integrals
use the second approach:
in \cite{cunningham-hales:good}, C. Cunningham and T.C. Hales prove that
\blue{when $\bG$ is unramified and}  
with $f$ is the characteristic function of $\fg(\ri_F)$,
$\cO_X$ is a motivic function on the set of so-called \emph{good} regular elements in $\fg(F)$. In fact, they prove a slightly more refined result, giving explicitly the residue field parameters that control the behaviour of this function.
In \cite{hales:orbital_motivic}, T.C. Hales proves that orbital integrals are motivic ``on average''.
Both of these papers pre-date the most general notion of a motivic function, and because of this, extra complications arise from having to explicitly track the parameters needed to define orbital integrals.

   
{In order to be able to use} the first approach to orbital integrals (as distributions), we 
   {define a notion of a ``motivic distribution'' (without this name, it first appears in this context in \cite{cluckers-cunningham-gordon-spice}*{\S 3})}. We say that a distribution $\Phi$ is motivic if    {it is, in fact, a collection of distributions defined for each $F\in \cC_{M'}$ for some $M'>0$, and} for any
   {suitable} family $(f_a)_{a\in S}$ of motivic test functions 
indexed by a parameter $a$ varying over a definable set $S$,
\blue{(where by ``suitable'' we mean that $f_{a, F}$ is in the domain of 
$\Phi$ for all $F\in \cC_M$ and  $a\in S_F$)}, 
there exists a motivic function {$g$} on $S$ {and an    {$M \ge M'$}} such that 
$\Phi(f_{a,F})=g(a)$ for all $F \in \cC_M$ and all    {$a \in S_F$,}
   

The most general current results on orbital integrals are:

\begin{enumerate}
\item Let $\bG$ be any connected reductive group (defined via a root datum $\xi$), with the Lie algebra $\fg$.  Let $\K$ be a local field of sufficiently large characteristic so that 
there exists a non-degenerate bilinear form $\langle, \rangle$ on $\fg(\K)$ as in 
\cite{adler-roche:intertwining}*{\S 4.1}. 
Every co-adjoint orbit in $\fg(\K)^\ast$ carries a volume form coming from the symplectic form defined by Kirillov (cf. 
\cite{kottwitz:clay}*{\S 17.3}), and the form $\langle, \rangle$ allows us to identify adjoint orbits with co-adjoint orbits. Let us denote the resulting volume form on the orbit of $X\in \fg(\K)$ by $\omega_X$. 
We prove that $(\omega_X)_{X\in \fg}$ is a family of definable volume forms indexed by $\fg$, see \cite{S-T}*{\S 1.5.3} and \cite{CGH-2}*{Proposition 4.2}. Thus,  

\emph {there exists a family of motivic measures on the orbits of all elements in the Lie algebra; in particular, with this choice of the normalization of the measures on adjoints orbits, all the orbital integrals form a family of motivic distributions}.	

However, in many situations one needs to use a specific normalization of the invariant measures on 
the orbits in the group or the Lie algebra. The results for the other normalizations are listed below. 


\item Cluckers, Hales and Loeser prove in \cite{cluckers-hales-loeser} that
for $\bG$ -- unramified over $F$, and with $f$ fixed to be the characteristic function of
$\fg(\ri_F)$,
$X\mapsto \cO_X(f)$ is a motivic function on the set of regular semisimple elements in $\fg(F)$,
where the measures on the orbits are the measures used in the Fundamental Lemma. 
In fact, their method shows also that the orbital    integrals of 
regular semisimple elements $X\in \fg(F)$ form a family of 
motivic distributions in the sense defined above. 



\item Still under the assumption that $\bG$ is unramified, in \cite{S-T}*{\S B.6}, we prove that if the centralizer $C_{\bG}(\gamma)$ of $\gamma\in \bG(F)$
splits over an unramified extension, our normalization of the measure
coincides with the canonical measure on the orbit that one gets from the
canonical measures (in the sense of \cite{gross:motive}) on $C_{\bG}(\gamma)(F)$ and $\bG(F)$; if $C_{\bG}(\gamma)$ is ramified, the motivic and canonical  measures might differ by a constant not exceeding a fixed power of $q$.
This result should easily generalize to the case when $\bG$ is ramified as well (though one more constant from the measure on $\bG(F)$ would appear).

We also note that even though we state the result for orbital integrals on the group, since this is what is required for the estimates that are the main goal of \cite{S-T}*{Appendix B}, in the course of the proof we essentially prove the analogous statement for the Lie algebra as well.

%\item It follows from the results of McNinch \cite{mcninch:nilpotent} and Jantzen \cite{jantzen} that the set of nilpotent elements $\mathcal N$ in the Lie algebra $\fg(F)$ is definable,
%cf \cite{CGH-2}*{\S 4.1}.
%We prove  in
%\cite{CGH-2}*{\S 4}
%that nilpotent orbital integrals (on the Lie algebra)    {form a family of} motivic distributions,    {indexed by the set of nilpotent elements $\mathcal N$}.

\item On the Lie algebra, one can consider the Fourier transform
$\widehat{\cO}_X(f)=\cO_X(\widehat f)$ of an orbital integral. By a theorem of Harish-Chandra
in the characteristic zero case, 
\blue{this distribution is represented by a locally constant function supported
on the set of regular elements $\fg(F)^{\reg}$. In positive characteristic,  Huntsinger \cite{adler-debacker:mk-theory} proved that the \emph{restriction of this distribution to the regular set} is represented by a locally constant function on 
$\fg(F)^{\reg}$}. 
We denote this function in both cases by
$\widehat{\mu}_X(Y)$, and think of it as a function on $\fg(F)$ that vanishes outside 
$\fg(F)^{\reg}$.

 In \cite{CGH-2}*{Theorem 5.8} we prove:
{\emph {There exists $M>0$ and a motivic function 
$h$ on $\fg\times \fg$ such that 
$\widehat \mu_X(Y) = h_F(X, Y)$, 
for all $F\in \cC_M$.
}}

\blue{The fact that the restriction of $\widehat{\mu}_X(Y)$ to the regular set is motivic allows us to transfer Harish-Chandra's result to positive characteristic; namely, we prove in \cite{CGH-2} that for \emph{any} $X\in \fg(F)$, and any 
$f\in C_c^\infty(\bG(F))$, one has $\mu_X(\hat f)=\int_{\fg(F)}\widehat{\mu}_X(Y) f(Y) dY$, provided the characteristic of $F$ is large enough. This result is discussed below 
in \S\ref{sec:locint}. }  

%\begin{enumerate}
%\item For nilpotent elements $X$, $\widehat \mu_X(Y)$ is a motivic
%function defined on $\mathcal N\times \fg$.
%\item For regular semisimple elements $X$,  $\widehat \mu_X(Y)$ is a motivic
%function defined on $\mathcal \fg^\reg\times \fg$.
%\end{enumerate}
\end{enumerate}

\subsection{Classification of nilpotent orbits,  and Shalika germs}
It follows from the results of McNinch \cite{mcninch:nilpotent} and Jantzen \cite{jantzen} that the set of nilpotent elements $\mathcal N$ in the Lie algebra $\fg(F)$ is definable.
%cf. \cite{CGH-2}*{\S 4.1}.

It is well-known that when the residue characteristic of the field $F$ is large enough,
there are finitely many nilpotent orbits in $\fg(F)$.
Using Waldspurger's parametrization of nilpotent orbits for classical Lie algebras
\cite{waldspurger:nilpotent}*{\S I.6}, it is possible to prove that these orbits are parametrized by the residue-field points of a definable set.
This approach is carried out in detail for odd orthogonal groups in \cite{diwadkar:o_n}.
Alternatively, one can use DeBacker's parametrization of nilpotent orbits that uses Bruhat-Tits theory, \cite{debacker:nilp}. This approach is demonstrated for the Lie algebra of type  $G_2$ (as well as for $\fsl_n$) in J. Diwadkar's thesis \cite{diwadkar:thesis}.

Using both parametrizations of nilpotent orbits, and the explicit matching between them
due to M. Nevins, \cite{nevins:param}, L. Robson proves
%the following statement in
in \cite{lance:thesis}
that for $\fsp_{2n}$, Shalika germs are motivic functions, up to a ``motivic constant''.
More precisely,    {the main result of \cite{gordon-robson} is}:
\begin{theorem} Let $\fg =\fsp_{2n}$. Let $\mathcal N$ be the set of nilpotent elements in $\fg$.  Then
\begin{enumerate}
\item There exists a definable set
$\mathcal E\subset {\rm RF}^n$ (that is, having only residue-field variables), and a definable  function $h:\mathcal N\to \mathcal E$, such that
for every $d\in \mathcal E$, $h^{-1}(d)$ is an adjoint orbit, and each orbit appears exactly once as the fibre of $h$.
\item There exist a finite collection of integers  ${(a_i)}_{i=0}^{r}$,  a motivic function $\Gamma$ on
$\mathcal E\times\fg^{\reg}$, and a constant $M>0$, such that for $F\in \cC_M$,
for every $d\in {\mathcal E}_F$, the function 
$\left(q^{a_0}\prod_{i=1}^r \frac 1{1-q^{a_i}}\right)^{   {-1}}\Gamma_F(d, \cdot)$ is a representative of the Shalika germ of the orbit $h^{-1}(d)$ on $\fg(F)^{\reg}$, \blue{in the sense that there exists a definable neighbourhood of zero on which it coincides with the canonical Shalika germ of the orbit $h^{-1}(d)$}.
\end{enumerate}
\end{theorem}

\blue{A refinement and generalization of this theorem to $\fsl_n$ and possibly non-classical Lie algebras is currently in progress.}

In a different context of groups over $\C\llp t\rrp $, a motivic interpretation of the
sub-regular germ is given in E. Lawes' thesis, \cite{lawes:thesis}.


\subsection{Transfer factors and other ingredients of the Fundamental Lemma}
In \cite{cluckers-hales-loeser}, the authors prove that transfer factors, weights, and other ingredients of the Fundamental Lemma are motivic functions. 
The same for the Fundamental Lemma of Jacquet-Rallis is proved in \cite{yun:fl-jr}. 


\subsection{Characters}
So far, the only results on Harish-Chandra characters of representations have been restricted to the cases where the character can be related to a semisimple orbital integral.
One of the fundamental problems is that the parametrization of supercuspidal representations via
$K$-types (and the classification of the $K$-types due to J.-K. Yu and Ju-Lee Kim) uses multiplicative characters of \blue{certain subgroups of $\bG(F)$, which in particular could be built from multiplicative characters of the field, and multiplicative characters} cannot be easily incorporated into the motivic integration framework. Because of this, at present we do not have a way of talking about representations in a completely field-independent way.
There are two partial results on characters of some representations in a suitable neighbourhood of the identity, where Murnaghan-Kirillov theory holds.
\begin{enumerate}
\item The main result of \cite{gordon:depth0} can be expressed in modern terms as:
\emph{the character of a \blue{uniform (i.e., coming from Deligne-Lusztig construction)} depth-zero representation of $\mathrm{Sp}_{2n}$ or
$\mathrm{SO}_{2n+1}$ is a motivic distribution (in the sense defined above) on the set of topologically unipotent elements}.
\item For positive-depth so-called \emph{toral, very supercuspidal} representations,
it is shown in  \cite{cluckers-cunningham-gordon-spice}, that up to a constant, their Harish-Chandra characters are motivic exponential functions on \emph{an explicitly described}  neighbourhood of the identity on which Murnaghan-Kirillov theory works, and an explicit parameter space (over the residue field) is constructed for these representations up to crude equivalence, where we identify two representations if their characters on this neighbourhood of the identity are the same.
\end{enumerate}

However, thanks to Harish-Chandra's local character expansion, extended to an expansion
near an arbitrary tame semisimple element by J. Adler and J. Korman 
\cite{adler-korman:loc-char-exp},
one can prove results about characters, using motivic integration, even  without
knowing that they are themselves motivic functions. We give an example of such an application in the next section.

We note that proving that the coefficients of Harish-Chandra's local character expansion
are motivic is a question of the same order of difficulty as the \blue{same} question about the characters  themselves  \blue{(at least, near the identity)}, \blue{since the Fourier transforms of orbital integrals are known to be motivic}, and so it is presently far from known.


\section{Local integrability results in positive characteristic}\label{sec:locint}
Following Harish-Chandra (and using the modification by R. Kottwitz
\cite{S-T}*{Appendix A} for non-regular elements),
for a semisimple element $\gamma\in \bG(F)$, we define the discriminant
$$D^G(\gamma):=
\prod_{\alpha\in \Phi\atop{ \alpha(\gamma)\neq 1}}|1-\alpha(\gamma)|, \quad \text{and}
\quad D(\gamma): =\prod_{\alpha\in \Phi}|1-\alpha(\gamma)|,
$$
where $\Phi$ is the root system of $\bG$.

On the Lie algebra, define
$D(X)=\prod_{\alpha\in \Phi}|\alpha(X)|$.
Following \cite{kottwitz:clay}, we call a function $h$ defined on $\fg(F)^\reg$
\emph{nice}, if it satisfies the following conditions:
\begin{itemize}
\item when extended by zero to
all of $\fg(\K)$, it is locally integrable, and
\item the function $D(X)^{1/2}h(X)$ is locally bounded on $\fg(\K)$.
\end{itemize}

Similarly, call a function on $\bG(\K)^\reg$ \emph{nice}, if it satisfies
the same conditions on $\bG(\K)$, with $D(X)$ replaced by its group version
$D(\gamma)$.

Combining the theorems of Harish-Chandra in characteristic zero with the Transfer Principles of \S \ref{sec:trpr} and the results of \S \ref{sub:oi} above,  we  obtain
\blue{the following theorem, which we state only in positive characteristic for emphasis, since we are using the characteristic zero version proved by Harish-Chandra in order to obtain it, and thus it says nothing new in the characteristic zero case}. 
\begin{thm}(cf. \cite{CGH-2}*{Theorem 2.1})\label{thm:orb int loc int}
For a connected reductive group $\bG$ (defined via specifying a root datum), there exists a constant $M_\bG>0$ that depends only on the absolute root datum of $\bG$,
such that
for every $\K\in \cB_{M_\bG}$, and every
adjoint orbit $\cO$ in $\fg(\K)$,
the function $\widehat \mu_{\cO}$  is a nice function on $\fg(\K)$.
\end{thm}


\blue{Our next result requires local character expansion in large positive characteristic; in characteristic zero  it is due to Harish-Chandra 
\cite{hc:queens}*{Theorem 16.2}, and  in \emph{large positive characteristic} it is 
proved by DeBacker \cite{debacker:homogeneity} near the identity,
and by Adler-Korman \cite{adler-korman:loc-char-exp} near a general tame semisimple element; we also need some weak information about the neighbourhood of validity of this expansion, which follows from the results of DeBacker and Adler-Korman.} 
These results require an additional hypothethis on the existence of the so-called mock exponential map $\mexp$, discussed in \cite{adler-korman:loc-char-exp}, which we will not quote here; we note only that for classical groups one can take $\mexp$ to be the Cayley transform.


\begin{thm}\label{thm:char}
Given an absolute root datum $\Psi$, there exists a constant $M>0$, such that for
every $F\in \cB_{M}$, every connected reductive group $\bG$ defined over $F$ with the absolute root datum $\Psi$, and every  admissible
representation $\pi$ of ${\bG}(F)$,
%Choose $r$ such that $\fg_r=\fg_{\rho(\pi)^+}$.
the Harish-Chandra character $\theta_{\pi}$ is a nice function on
$\bG(F)$, provided that mock exponential map $\mexp$ exists for $\bG(F)$.
\end{thm}

Finally, Theorem \ref{thm:orb int loc int} also implies (thanks to results of DeBacker, \cite{debacker:homogeneity}*{Theorem 2.1.5, Corollary 3.4.6, and Remark 2.1.7}) that Fourier transforms of general invariant distributions on $\fg(F)$ with support bounded modulo conjugation are represented by nice functions  in a neighbourhood of the origin. %This is Theorem \ref{thm:general}.



\section{Uniform bounds for orbital integrals}\label{sec:uni_b}

In order to state the next result we need to introduce the notation for a family
of test functions, namely, the generators of the spherical Hecke algebra. Assume that $\bG$ is an unramified reductive group with root datum $\xi$.
We have a Borel subgroup $B=TU$, and let $A$ be the maximal $F$-split torus in $T$.
Let  $K={\underline G}({\ri_F})$ be a hyperspecial maximal compact subgroup (where
${\underline G}$ is the smooth model associated with 
a hyperspecial point in an apartment of $C_{\bG}(A)$ by Bruhat-Tits theory).
For $\lambda\in X_\ast(A)$, let $\tau_\lambda^G$ be the characteristic function of
the double coset $K\lambda(\varpi)K$.



In \cite{S-T}*{Appendix B}, we prove, using Theorem~\ref{thm:presburger-fam},
\begin{thm}\label{thm:main2}
Consider an unramified root datum $\xi$. Then there exist constants $M>0$,  $a_\xi$ and $b_\xi$ that depend only on $\xi$,  such that for each non-Archimedean local field $F$
with residue characteristic at least $M$, the following holds. Let $\bG$ be a connected reductive algebraic group over
$F$ with the root datum $\xi$. {Let $A$ be a maximal $F$-split torus in $\bG$, and let $\tau_\lambda^G$ be as above.} Then
for all {$\lambda\in X_\ast(A)$} with $\|\lambda\|\le \kappa$,
$$
|\oi_\gamma(\tau_\lambda^G)|\le {q_F}^{a_\xi+b_\xi\kappa} D^G(\gamma)^{-1/2}$$
{for all semisimple elements $\gamma\in \bG(F)$.}
\end{thm}

It is sometimes useful to reformulate theorems of this type  to make an assertion about
a group defined over a global field. This is possible because given a connected reductive group
over a global field $\bF$, there are finitely many possibilities for the root data of 
   {$\bG_v=\bG\otimes_{\bF}\bF_v$}, as $v$ runs over the set of finite places of $\bF$, 
and    {$\bF_v$ denotes the completion of $\bF$ at $v$}.
Here we give an example of one such reformulation, which proved to be useful in \cite{S-T} 
({we also refer to \S 7 of loc.cit. for a more detailed explanation of the setting}).
\begin{thm}\label{thm:main}
Let $\bG$ be a connected reductive algebraic group over $\bF$, 
   {with a maximal split torus $A$}.
There exist constants $a_G$ and $b_G$ that depend only on the
global integral model of $\bG$ such that    {for
all but finitely many places $v$},
for all $\lambda\in X_\ast(A_v)$ with $\|\lambda\|\le \kappa$, 
$$
|\oi_\gamma(\tau_\lambda^G)|\le q_v^{a_G+b_G\kappa} D^G(\gamma)^{-1/2}$$
for all semisimple elements $\gamma\in \bG(\bF_v)$,
where $q_v$ is the cardinality of the residue field of $\bF_v$.
\end{thm}





\begin{bibdiv}
\begin{biblist}
\bibselect{references}
\end{biblist}
\end{bibdiv}


\end{document}




